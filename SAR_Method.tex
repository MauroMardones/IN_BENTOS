% Options for packages loaded elsewhere
\PassOptionsToPackage{unicode}{hyperref}
\PassOptionsToPackage{hyphens}{url}
\PassOptionsToPackage{dvipsnames,svgnames,x11names}{xcolor}
%
\documentclass[
]{article}
\usepackage{amsmath,amssymb}
\usepackage{iftex}
\ifPDFTeX
  \usepackage[T1]{fontenc}
  \usepackage[utf8]{inputenc}
  \usepackage{textcomp} % provide euro and other symbols
\else % if luatex or xetex
  \usepackage{unicode-math} % this also loads fontspec
  \defaultfontfeatures{Scale=MatchLowercase}
  \defaultfontfeatures[\rmfamily]{Ligatures=TeX,Scale=1}
\fi
\usepackage{lmodern}
\ifPDFTeX\else
  % xetex/luatex font selection
\fi
% Use upquote if available, for straight quotes in verbatim environments
\IfFileExists{upquote.sty}{\usepackage{upquote}}{}
\IfFileExists{microtype.sty}{% use microtype if available
  \usepackage[]{microtype}
  \UseMicrotypeSet[protrusion]{basicmath} % disable protrusion for tt fonts
}{}
\makeatletter
\@ifundefined{KOMAClassName}{% if non-KOMA class
  \IfFileExists{parskip.sty}{%
    \usepackage{parskip}
  }{% else
    \setlength{\parindent}{0pt}
    \setlength{\parskip}{6pt plus 2pt minus 1pt}}
}{% if KOMA class
  \KOMAoptions{parskip=half}}
\makeatother
\usepackage{xcolor}
\usepackage[margin=1in]{geometry}
\usepackage{color}
\usepackage{fancyvrb}
\newcommand{\VerbBar}{|}
\newcommand{\VERB}{\Verb[commandchars=\\\{\}]}
\DefineVerbatimEnvironment{Highlighting}{Verbatim}{commandchars=\\\{\}}
% Add ',fontsize=\small' for more characters per line
\usepackage{framed}
\definecolor{shadecolor}{RGB}{248,248,248}
\newenvironment{Shaded}{\begin{snugshade}}{\end{snugshade}}
\newcommand{\AlertTok}[1]{\textcolor[rgb]{0.94,0.16,0.16}{#1}}
\newcommand{\AnnotationTok}[1]{\textcolor[rgb]{0.56,0.35,0.01}{\textbf{\textit{#1}}}}
\newcommand{\AttributeTok}[1]{\textcolor[rgb]{0.13,0.29,0.53}{#1}}
\newcommand{\BaseNTok}[1]{\textcolor[rgb]{0.00,0.00,0.81}{#1}}
\newcommand{\BuiltInTok}[1]{#1}
\newcommand{\CharTok}[1]{\textcolor[rgb]{0.31,0.60,0.02}{#1}}
\newcommand{\CommentTok}[1]{\textcolor[rgb]{0.56,0.35,0.01}{\textit{#1}}}
\newcommand{\CommentVarTok}[1]{\textcolor[rgb]{0.56,0.35,0.01}{\textbf{\textit{#1}}}}
\newcommand{\ConstantTok}[1]{\textcolor[rgb]{0.56,0.35,0.01}{#1}}
\newcommand{\ControlFlowTok}[1]{\textcolor[rgb]{0.13,0.29,0.53}{\textbf{#1}}}
\newcommand{\DataTypeTok}[1]{\textcolor[rgb]{0.13,0.29,0.53}{#1}}
\newcommand{\DecValTok}[1]{\textcolor[rgb]{0.00,0.00,0.81}{#1}}
\newcommand{\DocumentationTok}[1]{\textcolor[rgb]{0.56,0.35,0.01}{\textbf{\textit{#1}}}}
\newcommand{\ErrorTok}[1]{\textcolor[rgb]{0.64,0.00,0.00}{\textbf{#1}}}
\newcommand{\ExtensionTok}[1]{#1}
\newcommand{\FloatTok}[1]{\textcolor[rgb]{0.00,0.00,0.81}{#1}}
\newcommand{\FunctionTok}[1]{\textcolor[rgb]{0.13,0.29,0.53}{\textbf{#1}}}
\newcommand{\ImportTok}[1]{#1}
\newcommand{\InformationTok}[1]{\textcolor[rgb]{0.56,0.35,0.01}{\textbf{\textit{#1}}}}
\newcommand{\KeywordTok}[1]{\textcolor[rgb]{0.13,0.29,0.53}{\textbf{#1}}}
\newcommand{\NormalTok}[1]{#1}
\newcommand{\OperatorTok}[1]{\textcolor[rgb]{0.81,0.36,0.00}{\textbf{#1}}}
\newcommand{\OtherTok}[1]{\textcolor[rgb]{0.56,0.35,0.01}{#1}}
\newcommand{\PreprocessorTok}[1]{\textcolor[rgb]{0.56,0.35,0.01}{\textit{#1}}}
\newcommand{\RegionMarkerTok}[1]{#1}
\newcommand{\SpecialCharTok}[1]{\textcolor[rgb]{0.81,0.36,0.00}{\textbf{#1}}}
\newcommand{\SpecialStringTok}[1]{\textcolor[rgb]{0.31,0.60,0.02}{#1}}
\newcommand{\StringTok}[1]{\textcolor[rgb]{0.31,0.60,0.02}{#1}}
\newcommand{\VariableTok}[1]{\textcolor[rgb]{0.00,0.00,0.00}{#1}}
\newcommand{\VerbatimStringTok}[1]{\textcolor[rgb]{0.31,0.60,0.02}{#1}}
\newcommand{\WarningTok}[1]{\textcolor[rgb]{0.56,0.35,0.01}{\textbf{\textit{#1}}}}
\usepackage{longtable,booktabs,array}
\usepackage{calc} % for calculating minipage widths
% Correct order of tables after \paragraph or \subparagraph
\usepackage{etoolbox}
\makeatletter
\patchcmd\longtable{\par}{\if@noskipsec\mbox{}\fi\par}{}{}
\makeatother
% Allow footnotes in longtable head/foot
\IfFileExists{footnotehyper.sty}{\usepackage{footnotehyper}}{\usepackage{footnote}}
\makesavenoteenv{longtable}
\usepackage{graphicx}
\makeatletter
\def\maxwidth{\ifdim\Gin@nat@width>\linewidth\linewidth\else\Gin@nat@width\fi}
\def\maxheight{\ifdim\Gin@nat@height>\textheight\textheight\else\Gin@nat@height\fi}
\makeatother
% Scale images if necessary, so that they will not overflow the page
% margins by default, and it is still possible to overwrite the defaults
% using explicit options in \includegraphics[width, height, ...]{}
\setkeys{Gin}{width=\maxwidth,height=\maxheight,keepaspectratio}
% Set default figure placement to htbp
\makeatletter
\def\fps@figure{htbp}
\makeatother
\setlength{\emergencystretch}{3em} % prevent overfull lines
\providecommand{\tightlist}{%
  \setlength{\itemsep}{0pt}\setlength{\parskip}{0pt}}
\setcounter{secnumdepth}{-\maxdimen} % remove section numbering
\newlength{\cslhangindent}
\setlength{\cslhangindent}{1.5em}
\newlength{\csllabelwidth}
\setlength{\csllabelwidth}{3em}
\newlength{\cslentryspacingunit} % times entry-spacing
\setlength{\cslentryspacingunit}{\parskip}
\newenvironment{CSLReferences}[2] % #1 hanging-ident, #2 entry spacing
 {% don't indent paragraphs
  \setlength{\parindent}{0pt}
  % turn on hanging indent if param 1 is 1
  \ifodd #1
  \let\oldpar\par
  \def\par{\hangindent=\cslhangindent\oldpar}
  \fi
  % set entry spacing
  \setlength{\parskip}{#2\cslentryspacingunit}
 }%
 {}
\usepackage{calc}
\newcommand{\CSLBlock}[1]{#1\hfill\break}
\newcommand{\CSLLeftMargin}[1]{\parbox[t]{\csllabelwidth}{#1}}
\newcommand{\CSLRightInline}[1]{\parbox[t]{\linewidth - \csllabelwidth}{#1}\break}
\newcommand{\CSLIndent}[1]{\hspace{\cslhangindent}#1}
\usepackage{fancyhdr}
\pagestyle{fancy}
\fancyhf{}
\lfoot[\thepage]{}
\rfoot[]{\thepage}
\fontsize{12}{22}
\selectfont
\ifLuaTeX
  \usepackage{selnolig}  % disable illegal ligatures
\fi
\IfFileExists{bookmark.sty}{\usepackage{bookmark}}{\usepackage{hyperref}}
\IfFileExists{xurl.sty}{\usepackage{xurl}}{} % add URL line breaks if available
\urlstyle{same}
\hypersetup{
  colorlinks=true,
  linkcolor={blue},
  filecolor={Maroon},
  citecolor={Blue},
  urlcolor={Blue},
  pdfcreator={LaTeX via pandoc}}

\title{\includegraphics[width=10cm,height=\textheight]{IEO-logo2.png}}
\author{}
\date{\vspace{-2.5em}}

\begin{document}
\maketitle


\pagenumbering{gobble}

%\begin{titlepage}
\begin{flushleft}
\Large{\textbf{Working Paper}}\\
\vspace*{2\baselineskip}
\LARGE{\textbf{Methodological implementation of Swept Area Ratio (SAR) in the wedge clam fishery \textit{Chamelea galllina} in the Gulf of Cádiz, Spain}}\\
\vspace*{5\baselineskip}
\Large{FEMP 04 Project}\\
\vspace*{1\baselineskip}
\Large{Instituto Español de Oceanografía, Cádiz }\\
\vspace*{4\baselineskip}
\end{flushleft}
\begin{flushright}
\large{\textit{Magro, Ana}}\\
\large{\textit{Mardones, Mauricio}}\\
\large{\textit{Delgado, Marina}}\\
\vspace*{1\baselineskip}
\normalsize{\textbf{Date}}\\
Abril, 2024
\end{flushright}

% \end{titlepage}


\hypersetup{linkcolor = black}
\newpage
\pagenumbering{roman}
%\tableofcontents
%\addcontentsline{toc}{section}{\contentsname}

\newpage



\pagenumbering{arabic}
\hypersetup{linkcolor = blue}

{
\hypersetup{linkcolor=}
\setcounter{tocdepth}{3}
\tableofcontents
}
\newpage

\hypertarget{abstract}{%
\section{ABSTRACT}\label{abstract}}

This abstract presents a method to calculate the SAR (Swept Area Ratio) using satellite data from a fleet of nearly 100 vessels in the artisanal fishing of Chirla in the Gulf of Cadiz. Artisanal fisheries, like Chirla fishing, often lack comprehensive data, hampering effective management. Leveraging satellite imagery, particularly through green boxes technology, offers a unique advantage by providing real-time spatial information on fishing activity. With this method, we can continuously identify the fleet's effort in both spatial and temporal terms. The objective is to develop an ad hoc approach reproducible through accessible codes and open data, facilitating sustainable fisheries management in the Gulf of Cadiz and beyond.

\hypertarget{introduction}{%
\section{INTRODUCTION}\label{introduction}}

Making better use of tracking data can reveal the spatiotemporal and intraspecific variability of species distributions

\hypertarget{methodology}{%
\section{METHODOLOGY}\label{methodology}}

\hypertarget{data-satelital-vessel}{%
\subsection{DATA SATELITAL VESSEL}\label{data-satelital-vessel}}

Por ahora solo trabajaremos con la data entregada por Candelaria Burgos para Chirla, que son los datos del año 2008 y 2009.

\hypertarget{results}{%
\subsection{RESULTS}\label{results}}

Identifico la cantidad de embarcaciones presentes en la BD

\begin{Shaded}
\begin{Highlighting}[]
\FunctionTok{unique}\NormalTok{(cajas2009uni}\SpecialCharTok{$}\NormalTok{MATRICULA)}
\end{Highlighting}
\end{Shaded}

De forma simple compruenbo los meses y dias con actividad de los datos sin filtrar;

Los métodos para la estimación y cartografiado del esfuerzo pesquero a partir de los datos de los VMS ya se han estudiado en varias pesquerías anteriormente. Estos estudios recomiendan un proceso en varios pasos según (\protect\hyperlink{ref-Cojan2012}{\textbf{Cojan2012?}}):

1- Borrar los registros duplicados,

2- Borrar los registros en puertos.

3- Calcular el intervalo de tiempo entre registros sucesivos,

La idea es identificar los registros con tiempo efectivo de arrastre como lo muestra la Figura \ref{fig:esq};

De esta forma, a cada señal proporcionada por la caja verde se le asignó una actividad: pesca, maniobra o navegada. En la figura 8 se puede ver un histograma que representa el número de registros en función de la velocidad para los registros filtrados, en él se observa cómo han desaparecido los registros en puerto y que existen tres modas correspondientes a las actividades mencionadas, maniobras (M), pesca (P) y navegaciones (N). (\protect\hyperlink{ref-Cohan2012}{Cojan, 2012})

Los registros en los cuales la velocidad del buque fue inferior a 1.5 nudos o entre 3.5 y 6 nudos, fueron considerados como maniobras de pesca (actividad ``M''), tales como la virada y largada del arte o el reposicionamiento del buque precedente al arrastre.

4- Ahora calculo las distancias entre puntos. como??

\newpage

\hypertarget{muxe9todo-1}{%
\subsubsection{Método 1}\label{muxe9todo-1}}

\begin{Shaded}
\begin{Highlighting}[]
\NormalTok{calcular\_DISTANCIA }\OtherTok{\textless{}{-}} \ControlFlowTok{function}\NormalTok{(lat1, lon1, lat2, lon2) \{}
  \CommentTok{\# Converto to radians}
\NormalTok{  lat1\_rad }\OtherTok{\textless{}{-}}\NormalTok{ lat1 }\SpecialCharTok{*}\NormalTok{ pi }\SpecialCharTok{/} \DecValTok{180}
\NormalTok{  lon1\_rad }\OtherTok{\textless{}{-}}\NormalTok{ lon1 }\SpecialCharTok{*}\NormalTok{ pi }\SpecialCharTok{/} \DecValTok{180}
\NormalTok{  lat2\_rad }\OtherTok{\textless{}{-}}\NormalTok{ lat2 }\SpecialCharTok{*}\NormalTok{ pi }\SpecialCharTok{/} \DecValTok{180}
\NormalTok{  lon2\_rad }\OtherTok{\textless{}{-}}\NormalTok{ lon2 }\SpecialCharTok{*}\NormalTok{ pi }\SpecialCharTok{/} \DecValTok{180}
  \CommentTok{\# Ratio earth in mts}
\NormalTok{  radio\_tierra }\OtherTok{\textless{}{-}} \DecValTok{6371000}
  \CommentTok{\# Calculate distance using haversine equation}
\NormalTok{  dlat }\OtherTok{\textless{}{-}}\NormalTok{ lat2\_rad }\SpecialCharTok{{-}}\NormalTok{ lat1\_rad}
\NormalTok{  dlon }\OtherTok{\textless{}{-}}\NormalTok{ lon2\_rad }\SpecialCharTok{{-}}\NormalTok{ lon1\_rad}
\NormalTok{  a }\OtherTok{\textless{}{-}} \FunctionTok{sin}\NormalTok{(dlat }\SpecialCharTok{/} \DecValTok{2}\NormalTok{)}\SpecialCharTok{\^{}}\DecValTok{2} \SpecialCharTok{+} \FunctionTok{cos}\NormalTok{(lat1\_rad) }\SpecialCharTok{*} \FunctionTok{cos}\NormalTok{(lat2\_rad) }\SpecialCharTok{*} \FunctionTok{sin}\NormalTok{(dlon }\SpecialCharTok{/} \DecValTok{2}\NormalTok{)}\SpecialCharTok{\^{}}\DecValTok{2}
\NormalTok{  c }\OtherTok{\textless{}{-}} \DecValTok{2} \SpecialCharTok{*} \FunctionTok{atan2}\NormalTok{(}\FunctionTok{sqrt}\NormalTok{(a), }\FunctionTok{sqrt}\NormalTok{(}\DecValTok{1} \SpecialCharTok{{-}}\NormalTok{ a))}
\NormalTok{  DISTANCIA }\OtherTok{\textless{}{-}}\NormalTok{ radio\_tierra }\SpecialCharTok{*}\NormalTok{ c}
  \FunctionTok{return}\NormalTok{(DISTANCIA)}
\NormalTok{\}}
\CommentTok{\# transform date  POSIXct con lubridate}
\NormalTok{datos}\SpecialCharTok{$}\NormalTok{fecha\_hora }\OtherTok{\textless{}{-}} \FunctionTok{ymd\_hms}\NormalTok{(}\FunctionTok{paste}\NormalTok{(datos}\SpecialCharTok{$}\NormalTok{FECHA, datos}\SpecialCharTok{$}\NormalTok{HORA))}

\CommentTok{\# orden DF in arrange by time}
\NormalTok{datos }\OtherTok{\textless{}{-}}\NormalTok{ datos }\SpecialCharTok{\%\textgreater{}\%} 
  \FunctionTok{arrange}\NormalTok{(fecha\_hora)}
\CommentTok{\# estimate disntance by row}
\NormalTok{datos}\SpecialCharTok{$}\NormalTok{DISTANCIA }\OtherTok{\textless{}{-}} \FunctionTok{mapply}\NormalTok{(calcular\_DISTANCIA,}
\NormalTok{                          datos}\SpecialCharTok{$}\NormalTok{N\_LATITUD,}
\NormalTok{                          datos}\SpecialCharTok{$}\NormalTok{N\_LONGITUD,}
                          \FunctionTok{lag}\NormalTok{(datos}\SpecialCharTok{$}\NormalTok{N\_LATITUD),}
                          \FunctionTok{lag}\NormalTok{(datos}\SpecialCharTok{$}\NormalTok{N\_LONGITUD))}

\CommentTok{\# firts row is NA because  is firts register}
\NormalTok{datos}\SpecialCharTok{$}\NormalTok{DISTANCIA[}\DecValTok{1}\NormalTok{] }\OtherTok{\textless{}{-}} \ConstantTok{NA}
\FunctionTok{head}\NormalTok{(datos)}
\end{Highlighting}
\end{Shaded}

\hypertarget{muxe9todo-2}{%
\subsubsection{Método 2}\label{muxe9todo-2}}

Probar forma que indica Ana Magro, que es calcular tiempo de arrastre X Velocidad. La idea es calcular el tiempo entre registros dado por \texttt{fecha\_hora}. Y multiplicamos por \texttt{N\_VELOCIDAD}

5- Diferenciar entre registros de pesca y no pesca basándose en la velocidad y solo dejó los registros \texttt{P}

ahora dejo valores de distancia menores a 1 km (preguntar). Luego calculo las variables de velocidad en metros/seg. y tiempo recorrido por la rastra.

Gafico velocidad y SA promedio por barco

Calculo el SAR

De acuerdo a Church et al. (\protect\hyperlink{ref-Church2016}{2016}), el cálculo de la Razón del Área Barrida (Swept Area Ratio, SAR) \texttt{SA} es el área barrida (mts/2), \texttt{CA} es el área de la celda y \texttt{SAR} es la proporción del área barrida (equivalente al número de veces que la celda fue barrida).

donde;

\[
SAr = \frac{SA}{CA}
\]

donde \texttt{SA}sera la distancia recorrida el arrastre y la apertura en metros del draga, es decir;

\[
SA = Distancia \times Apertura \ Draga
\]
Pero primero, debemos engrillar la data y luego calcular por cada celda

Ahora produzco un mapa de las grillas utilizadas en la pesquería de Chirla. Estos datos vectoriales fueron obtenidos desde la paina oficial de datos espaciales de la Junta de Andalucia \href{https://portalrediam.cica.es/descargas?path=\%2F08_AMBITOS_INTERES_AMBIENTAL\%2F02_LITORAL_MARINO\%2F04_SOCIOECONOMIA\%2FZonasProduccionMoluscos}{Shapesfile}

\hypertarget{leo-shapes-y-transformo-a-la-proyecciuxf3n-correcta.}{%
\subsection{Leo Shapes y transformo a la proyección correcta.}\label{leo-shapes-y-transformo-a-la-proyecciuxf3n-correcta.}}

Mapa test

Ahora identifico la base que quiero plotear y hago el calculo de \texttt{SAR}

This grid has the same characteristics as the environmental data grids
that will be called up later. This grid is 1x0.5 degrees which allows a
clear visualization of the processes, whether biological and/or
environmental.

Y ahora veo como es el ttotal del SAR en \% por celda

ploteo SAR

Ahora trato de engrillar los habitat

Priebo el mapa

Cuento cuantas estaciones hay por habitat.

\hypertarget{calculo-de-lances-por-estacion-o-por-habitat}{%
\subsection{Calculo de lances por estacion o por habitat}\label{calculo-de-lances-por-estacion-o-por-habitat}}

Pero para esto, primer debemos asignar un ID para cada operación.

Solo para visualizar, cambio el objeto a \texttt{data.frame}. Pero debo considerar \texttt{idlance4} para engrillar.

\hypertarget{discussion}{%
\section{DISCUSSION}\label{discussion}}

\hypertarget{conclusion}{%
\section{CONCLUSION}\label{conclusion}}

\newpage

\hypertarget{referencias}{%
\section*{REFERENCIAS}\label{referencias}}
\addcontentsline{toc}{section}{REFERENCIAS}

\hypertarget{refs}{}
\begin{CSLReferences}{1}{0}
\leavevmode\vadjust pre{\hypertarget{ref-Church2016}{}}%
Church, N. J., Carter, A. J., Tobin, D., Edwards, D., Eassom, A., Cameron, A., Johnson, G. E., Robson, L. M., \& Webb, K. E. (2016). {JNCC Pressure Mapping Methodology. Physical Damage (Reversible Change)-Penetration and/or disturbance of the substrate below the surface of the seabed, including abrasion}. \emph{JNCC Report No}, \emph{515}(December).

\leavevmode\vadjust pre{\hypertarget{ref-Cohan2012}{}}%
Cojan, M. (2012). \emph{{AN{Á}LISIS Y SEGUIMIENTO DE LA FLOTA DE DRAGAS HIDR{Á}ULICAS EN EL GOLFO DE C{Á}DIZ}} (pp. 1--23) {[}PhD thesis{]}.

\end{CSLReferences}

\end{document}
